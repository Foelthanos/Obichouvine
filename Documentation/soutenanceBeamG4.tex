% !TEX TS-program = pdflatex
% !TEX encoding = UTF-8 Unicode

% This file is a template using the "beamer" package to create slides for a talk or presentation
% - Giving a talk on some subject.
% - The talk is between 15min and 45min long.
% - Style is ornate.

% MODIFIED by Jonathan Kew, 2008-07-06
% The header comments and encoding in this file were modified for inclusion with TeXworks.
% The content is otherwise unchanged from the original distributed with the beamer package.

\documentclass{beamer}


% Copyright 2004 by Till Tantau <tantau@users.sourceforge.net>.
%
% In principle, this file can be redistributed and/or modified under
% the terms of the GNU Public License, version 2.
%
% However, this file is supposed to be a template to be modified
% for your own needs. For this reason, if you use this file as a
% template and not specifically distribute it as part of a another
% package/program, I grant the extra permission to freely copy and
% modify this file as you see fit and even to delete this copyright
% notice. 


\mode<presentation>
{
  \usetheme{Berkeley}
  % or ...

  \setbeamercovered{transparent}
  % or whatever (possibly just delete it)
}


\usepackage[english]{babel}
% or whatever

\usepackage[utf8]{inputenc}
% or whatever

\usepackage{times}
\usepackage[T1]{fontenc}
% Or whatever. Note that the encoding and the font should match. If T1
% does not look nice, try deleting the line with the fontenc.


\title[UE PROG6] % (optional, use only with long paper titles)
{Soutenance de projet}

\subtitle
{Interface conviviale pour le jeu du Tablut} % (optional)

\author[] % (optional, use only with lots of authors)
{B.~Berkati \and E.~Berthier \and A.~Canonne \and M.~Dufrenoy \and M.~Duplan \and L.~Postic}
% - Use the \inst{?} command only if the authors have different
%   affiliation.

\institute[Universities of Somewhere and Elsewhere] % (optional, but mostly needed)
{
  IM2AG - Département STS Informatique\\
  Université Joseph Fourier
 }
% - Use the \inst command only if there are several affiliations.
% - Keep it simple, no one is interested in your street address.

\date[UET Animation scientifique] % (optional)
{6 juin 2014 - IM2AG}

\subject{Talks}
% This is only inserted into the PDF information catalog. Can be left
% out. 



% If you have a file called "university-logo-filename.xxx", where xxx
% is a graphic format that can be processed by latex or pdflatex,
% resp., then you can add a logo as follows:

% \pgfdeclareimage[height=0.5cm]{university-logo}{university-logo-filename}
% \logo{\pgfuseimage{university-logo}}



% If you wish to uncover everything in a step-wise fashion, uncomment
% the following command: 

%\beamerdefaultoverlayspecification{<+->}


\begin{document}

\begin{frame}
  \titlepage
\end{frame}




% Since this a solution template for a generic talk, very little can
% be said about how it should be structured. However, the talk length
% of between 15min and 45min and the theme suggest that you stick to
% the following rules:  

% - Exactly two or three sections (other than the summary).
% - At *most* three subsections per section.
% - Talk about 30s to 2min per frame. So there should be between about
%   15 and 30 frames, all told.

\section{Introduction}

\begin{frame}{Introduction}
  \tableofcontents
  % You might wish to add the option [pausesections]
\end{frame}

\section{Contexte du projet}

\subsection{Jeu du tablut}

\begin{frame}{Jeu de Tafl}
  \begin{itemize}
  \item Un tablier de jeu carré
	\item Des cases marquées symétriquement
	\item Des forces inégales
	\item Un principe de prise par encadrement
	\item Des objectifs de jeux différents
	\item Une pièce particulière, le Roi
  \end{itemize}
\end{frame}

\subsection{Contexte du projet}

\section{Réalisation technique globale}
\begin{frame}{Réalisation technique globale}{Le scénario}

\begin{columns}

\begin{column}{4cm}
	Partie 1 :
	\begin{itemize}
		\item Objectif : Motiver les élèves
		\item Utilisation d'un jeu par équipe
		\item Fil rouge : Enigma
		\item Conclusion : Logique, flexibilité et rigueur
	\end{itemize}
\end{column}

\begin{column}{6cm}
	Partie 2 :
	\begin{itemize}
		\item Objectif : Expliquer l'utilité des mathématiques
		\item Utilisation d'exemples concrets : Informatique, cuisine, conduite
		\item Conclusion : Vous utilisez et utiliserez ces compétences toute votre vie
	\end{itemize}
\end{column}

\end{columns}

\end{frame}

\section{IHM}

\section{IA}


\section{Bilan}

\begin{frame}{Bilan}

  % Keep the summary *very short*.
  \begin{itemize}
  \item Les élèves ont parfaitemet integrer les concepts lors du jeu
  \item Certains élèves ont pris conscience de la place des mathématiques dans leur vie
  \item Un groupe a réussi à recréer un chiffre de Vigenère et à l'utiliser !
  \end{itemize}
  
 
\end{frame}


\end{document}


